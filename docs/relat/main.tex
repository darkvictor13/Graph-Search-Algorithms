\documentclass[12pt, a4paper]{article}
\usepackage{meu}
\usepackage{booktabs}
\usepackage{siunitx}

\renewcommand{\thefootnote}{\roman{footnote}}
\setlength{\headheight}{15pt}

\begin{document}
\capa%
\tableofcontents\cleardoublepage%
%\listoffigures\cleardoublepage
%\listoftables\cleardoublepage

\section{Introdução}\label{sec:intro}
O Projeto tem como objetivo a implementação de um sistema para auxiliar turistas em Veneza a chegarem aos diversos museus da cidade.

\section{Implementação}\label{sec:impl}

\subsection{Algoritmos de busca}\label{sec:algoritmos}
\subsubsection{Algoritmo A* (melhor solução)}\label{sec:astar}
\subsubsection{Busca em largura (pior solução)}\label{sec:bl}
\subsubsection{Busca em profundidade (extra)}\label{sec:bp}

\section{Resultados}\label{sec:resultados}
Com o sistema implementado, foi possível realizar os seguintes testes:
\subsection{Descrição do teste}
\begin{itemize}
    %TODO: Definir X
    \item Para cada um dos algoritmos de busca implementados,
    foi executado X vezes passando os parâmetros \( arquivo ~ saida_i ~ algoritmos_j \),
    sendo \( i \in \{1, \ldots, X\} \) e \( j \in \{ \)A*,\( \} \).
    \item Os testes foram realizados em um computador com as seguintes características:
    \begin{itemize}
        \item\textbf{Processador}: i3--1115G4 4.100GHz;
        \item\textbf{Memória RAM}: 8GB\@;
        \item\textbf{SSD}: 256GB\@;
        \item\textbf{Sistema operacional}: Arch Linux~\footnote{Link para download do sistema operacional\url{https://archlinux.org/download/}}.
    \end{itemize}
    \item A execução dos testes foi realizada logo após a inicialização do computador, sem nenhum outro processo de usuário em execução.
\end{itemize}

\subsection{Resultados dos testes}\label{sec:res}


\section{Conclusão}\label{sec:concl}

%\bibliography{ref}

\end{document}
