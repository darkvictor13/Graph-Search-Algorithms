\documentclass[12pt, a4paper]{article}
\usepackage{meu}
\usepackage{booktabs}
\usepackage{siunitx}

\renewcommand{\thefootnote}{\roman{footnote}}
\setlength{\headheight}{15pt}

\begin{document}
\capa%
\tableofcontents\cleardoublepage%
%\listoffigures\cleardoublepage
%\listoftables\cleardoublepage

\section{Introdução}\label{sec:intro}
O Projeto tem como objetivo a implementação de um sistema para auxiliar turistas em Veneza a chegarem aos diversos museus da cidade.

Implementado inteiramente em c++20 sem utilização de bibliotecas externas além das bibliotecas padrão do c++20,
podendo assim ser compilado em qualquer sistema operacional que possua o compilador adequado.

Essa linguagem foi escolhida pois além de possuir abstrações alto nível com o uso de classes,
ainda é extremamente eficiente ao ser compilada diretamente para para linguagem de máquina.
Além do disso a dupla responsável pelo projeto possui familiaridade com a linguagem.

O sistema é capaz de calcular a rota mais curta entre dois pontos,
utilizando os algoritmos A*, busca em largura e busca em profundidade.

Visando a simplicidade e levar os algoritmos a seus limites de eficiência,
o sistema não implementa uma interface gráfica,
todos os argumentos devem ser passados como argumentos de linha de comando,
ou coletados em tempo de execução, para ver os argumentos para o programa veja a seção~\ref{sec:executando}.

Uma vez que todos os parâmetros para execução do programa podem ser passados antes que ele inicie,
pode-se fazer uso de scripts para automatizar a execução do programa gerando resultados para diferentes cenários.
Dessa maneira foi implementado um script para testar o desempenho, explicado na seção~\ref{sec:resultados}.

\section{Executando o programa}\label{sec:executando_programa}
\subsection{Requisitos}\label{sec:requisitos}
\begin{itemize}
    \item cmake;
    \item make;
    \item g++;
    \item git (opcional).
\end{itemize}

\subsection{Compilando}\label{sec:compilando}
\begin{itemize}
    \item\texttt{cd build}
    \item\texttt{cmake .}
    \item\texttt{make}
\end{itemize}

\subsection{Executando}\label{sec:executando}
Para executar o programa basta executar o arquivo

\texttt{Graph\_Search\_Algorithms\_1.0.0} gerado na pasta build.

Caso nada seja passado como argumento, o programa irá coletar os argumentos em tempo de execução,
sendo eles:
\begin{enumerate}
    \item O caminho para o arquivo de entrada;
    \item O caminho para o arquivo de saída contendo o resumo da execução;
    \item O algoritmo a ser utilizado, podendo ser:
    \begin{itemize}
        \item A \( \rightarrow \) A*;
        \item B \( \rightarrow \) \textit{BFS, Breadth First Search} (busca em largura);
        \item D \( \rightarrow \) \textit{DFS, Depth First Search} (busca em profundidade).
    \end{itemize}
\end{enumerate}

\section{Implementação}\label{sec:impl}

\subsection{Estruturas de dados}\label{sec:estruturas}

\subsection{Algoritmos de busca}\label{sec:algoritmos}
\subsubsection{Algoritmo A* (melhor solução)}\label{sec:astar}
\subsubsection{Busca em largura (pior solução)}\label{sec:bl}
\subsubsection{Busca em profundidade (extra)}\label{sec:bp}

\section{Resultados}\label{sec:resultados}
Com o sistema implementado, foi possível realizar os seguintes testes:
\subsection{Descrição do teste}
\begin{itemize}
    %TODO: Definir X
    \item Para cada um dos algoritmos de busca implementados,
    foi executado X vezes passando os parâmetros \( arquivo ~ saida_i ~ algoritmos_j \),
    sendo \( i \in \{1, \ldots, X\} \) e \( j \in \{ \)A*,\( \} \).
    \item Os testes foram realizados em um computador com as seguintes características:
    \begin{itemize}
        \item\textbf{Processador}: i3--1115G4 4.100GHz;
        \item\textbf{Memória RAM}: 8GB\@;
        \item\textbf{SSD}: 256GB\@;
        \item\textbf{Sistema operacional}: Arch Linux~\footnote{Link para download do sistema operacional\url{https://archlinux.org/download/}}.
    \end{itemize}
    \item A execução dos testes foi realizada logo após a inicialização do computador, sem nenhum outro processo de usuário em execução.
\end{itemize}

\subsection{Resultados dos testes}\label{sec:res}


\section{Conclusão}\label{sec:concl}

%\bibliography{ref}

\end{document}
